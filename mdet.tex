\documentclass[fleqn,useAMS,usenatbib]{mnras}
%=====================================================================
% CUSTOM: PACKAGES, MACROS & SETTINGS
%=====================================================================
% packages for figures
\usepackage{graphicx,todonotes}

% packages for symbols
\usepackage{latexsym,amssymb}

% AMS-LaTeX package for e.g. subequations
\usepackage{amsmath,morefloats}
\usepackage{natbib,graphicx,amsmath,subfigure,color,xcolor,hyperref}

\topmargin-1cm

\graphicspath{{figures/}}

\newcommand\notedo[1]{\todo[color=yellow, inline, size=\small]{To do:#1}}
\newcommand\notewrite[1]{\todo[color=orange, inline, size=\small]{To write: #1}}
\newcommand\noteask[1]{\todo[color=cyan, inline, size=\small]{To ask: #1}}
\newcommand\notecontrib[1]{\todo[color=green, inline, size=\small]{Contributors: #1}}
\newcommand\esstodo[1]{\todo[color=yellow, inline, size=\small]{ESS: #1}}
\newcommand{\ess}[1]{\textcolor{red}{[ESS: \bf #1]}}
\newcommand{\mrb}[1]{\textcolor{purple}{[MRB: \bf #1]}}


\newcommand{\vecg}{\mbox{\boldmath $g$}}
\newcommand{\vece}{\mbox{\boldmath $e$}}
\newcommand{\veck}{\mbox{\boldmath $k$}}
\newcommand{\vecQ}{\mbox{\boldmath $Q$}}
\newcommand{\vecF}{\mbox{\boldmath $F$}}
\newcommand{\vecD}{\mbox{\boldmath $D$}}
\newcommand{\matR}{\mbox{$\bf R$}}
\newcommand{\matC}{\mbox{$\bf C$}}
\newcommand{\bnab}{\boldsymbol{\nabla}}
\newcommand{\bnabg}{\boldsymbol{\nabla_g}}
\newcommand{\galsim}{\texttt{GALSIM}}
\newcommand{\ngmix}{\texttt{ngmix}}
\newcommand{\nnsim}{\texttt{nsim}}
\newcommand{\snr}{$S/N$}
\newcommand{\sn}{$S/N$}
\newcommand{\coadd}{{\rm coadd}}
\newcommand{\desreq}{$4\times 10^{-3}$}
\newcommand{\lsstreq}{$2\times 10^{-3}$}

\newcommand{\mcal}{\textsc{metacalibration}}
\newcommand{\mdet}{\textsc{metadetection}}
\newcommand{\Mcalshort}{\textsc{metacal}}
\newcommand{\Mcal}{\textsc{Metacalibration}}
\newcommand{\Mdet}{\textsc{Metadetection}}
\newcommand{\vest}{\mbox{\boldmath $e$}}
\newcommand{\est}{e}
\newcommand{\mcalR}{\mbox{\boldmath $R$}}
\newcommand{\mcalRS}{\mbox{\boldmath $R_S$}}
\newcommand{\gest}{\mbox{\boldmath $\hat \gamma$}}
\newcommand{\vecgam}{\mbox{\boldmath $\gamma$}}

\newcommand{\sx}{\textsc{SExtractor}}

\newcommand{\bfd}{\textsc{BFD}}

\newcommand{\vonkarman}{{von K\'arm\'an}~}

\title[\Mdet]{\Mdet: Mitigating Shear-dependent Object Detection Biases with \Mcal}

\author[Sheldon et~al.]{Erin Sheldon$^1$, Matthew R. Becker$^2$,
Niall MacCrann$^{3,4}$, Michael Jarvis$^5$
  \\$^1$Brookhaven National Laboratory, Bldg. 510, Upton, NY 11973, USA
  \\$^2$High Energy Physics Division, Argonne National Laboratory, Lemont, IL 60439, USA
  \\$^3$Center for Cosmology and Astro-Particle Physics, The Ohio State University, Columbus, OH 43210, USA
  \\$^4$Department of Physics, The Ohio State University, Columbus, OH 43210, USA
  \\$^5$Department of Physics and Astronomy, University of Pennsylvania, Philadelphia, PA 19104, USA
}

\begin{document}
\date{Draft \today}
\maketitle

\begin{abstract}

\Mcal\ is a new technique for measuring weak gravitational lensing shear that
is unbiased for isolated galaxy images.  In this work we test the \mcal\ method
with overlapping, or ``blended'' galaxy images.  Using standard \mcal\ we find
a significant shear calibration bias when objects overlap. This bias is a few
percent for galaxy densities relevant for current surveys, and increases to
nearly ten percent for future surveys.  This bias is due to shear-dependent
object detection rather than object blending itself; if object detection is shear
independent, no de-blending of images is needed, in principle.
We demonstrate that shear-dependent object detection biases are accurately
removed when including object detection in the \mcal\ process, a technique we call
\mdet. Finally, we discuss the physical limits of \mdet's accuracy and address one
of the primary barriers to applying \mdet\ to realistic data, the variation of the
point-spread function across the image.

\end{abstract}

\section{Introduction}


Recently introduced methods for the estimation of weak gravitational lensing
shear promise to provide calibration at the 0.1\% level, adequate for the
requirements of future weak lensing surveys.  Two methods that have
demonstrated sufficient accuracy without the use of calibration from
simulations, and without any compromise of precision, are the \bfd\ method
\citep{BernBFD2016} and the \mcal\ method \citep{HuffMcal2017,SheldonMcal2017}.
The tests of these methods, while stringent, did not include an important
aspect of the real universe: the images of objects overlap on the sky. In this
work we test \mcal\ in the presence of blending, and demonstrate that there is
a bias associated with shear-dependent object detection. We introduce an improvement
to the method that naturally accounts for this shear-dependent object detection bias.

A method such as \mcal\ can in principle calibrate any measurement biases, even
those associated with blending \citep[e.g.,][]{DawsonBlending2016}. More
problematic for shear calibration is the dependence of the object detection on
shear.\footnote{In this work, we will use the terms "detection" and "object detection"
interchangeably to mean the detection of an object above threshold. This process
is distinct but related to the detection of flux above threshold in an image.
The detection of objects typically includes a step that processes pixels with
detected flux into a catalog of objects in the image.} In the weak shear regime, the lensing mapping is simple shear
transformation that preserves surface brightness \citep{SchneiderBook92}. For
such a simple mapping, object detection need not be shear dependent. For example, an
object detection algorithm that simply identifies connected regions above a threshold
as a single object will not in principle be shear dependent. However, in real
observations the image is smeared by a point-spread-function (PSF) due to the
atmosphere, telescope optics or detector. In this case the overlap of objects
does depend on shear because the PSF convolution occurs after the shear mapping.
In this case the simple threshold objct detection method mentioned above will manifest
a shear-dependent object detection bias. This effect is demonstrated in figure
\ref{fig:toy}.


\begin{figure*}
  \includegraphics[width=\textwidth]{figures/toy.png}

  \caption{ Toy example of shear-dependent object detection in the presence of a
  PSF.  In panel (a) two objects are present, convolved by a PSF with no
  shear.  Contours represent constant brightness.  In panel (b) the objects
  are sheared by $\gamma = (0.0, 0.1)$ {\em after} the PSF convolution.
  Contours are the same as panel (a).  In this case the inner contours for
  the two objects overlap before and after application of the shear.  This is
  a general property of the shear transformation in the weak regime. In panel
  (c) the shear is applied {\em before} the PSF convolution, which mimics real
  sky images. In this case the inner contours do not overlap after shearing.
  For case (c) an object detection algorithm that identified connected regions above
  a threshold as a single object would manifest a shear-dependent object detection
  bias.  \label{fig:toy} }

\end{figure*}

Common object detection schemes in use today, such at those in Source Extractor
\citep{Bertin96} and the HST/LSST pipelines \citep{BoschHSC2018,BoschLSST2018}
are based on thresholding, similar to the simple approach described above but
differing in complexity and efficiency. Thus we would expect object detections
produced by those codes to also manifest shear dependence. An open question,
which we will not address in this work, is whether or not one can derive an object
detection algorithm which is independent of the shear applied to the image
in the presence of PSF and detector distortions. Such an algorithm would
in principle eliminate the shear-dependent object detection biases explored in this
work.

Current implementations of \mcal\ \citep[e.g.,][]{HuffMcal2017,SheldonMcal2017},
when used with the common object detection schemes discussed above, are expected to
exhibit shear-dependent object detection biases. These implementations of \mcal\ work
by applying artificial shears to small postage stamp images for objects found in
a {\em preexisting, independent} object detection step. This object detection step has the
shear-dependent object detection bias in it already, and thus any shear applied when
running \mcal\ does not properly measure the full shape response to shear. Note
that corrections for selection effects were derived in \cite{SheldonMcal2017},
but those do not work near the object detection threshold since objects needed for the
corrections may not have been found at all. As we will demonstrate, object detection
effects can be incorporated naturally by shearing larger images, rather than
small postage stamps, and re-running the object detection algorithm on each of the
sheared images. In this process, shear measurement errors, selection effects,
object detection, and any possible blending effects are all accounted for.
We call this technique \mdet.

In this work, we focus on elucidating the source of shear-dependent object detection
biases and the performance of the core \mdet\ algorithm. We demonstrate that
close blends, which are detected as only one object roughly half of the time,
exhibit the strongest biases. Further, we provide evidence that object
deblending techniques that are based on the observed set of objects cannot
correct for the biases. Next, we demonstrate that \mdet\ can produce unbiased
shear estimates in simulations with realistic, LSST-like levels of object
blending. After that, we estimate the impact of the main physical assumption
made by \mdet, namely that all objects in a small region of the sky are at the
same redshift and experience the same shear. Finally, we begin to address some
of the technical challenges of implementing \mdet\ on real imaging data by
demonstrating that the technique, when applied to multi-epoch imaging surveys,
can naturally average out enough of the PSF variation.

Future work will need to continue to address a number of technical challenges associated
with implementing \mdet\ on real data. First, beyond PSF variaton, larger images
can have non-trivial world coordinate system (WCS) variation. Usually these observational
details are approximately constant over small post-stamps and so can be dealt
with easily. Second, larger regions of images can have non-trivial masking
patterns due to stellar diffraction spikes, cosmic rays, etc. In the
implementation presented in this work, we use fast Fourier transforms (FFTs) to
handle convolutions, so that this data needs to be interpolated in some way.
Finally, \mdet\ catalogs will need to be incorporated into the full set of
downstream analysis tasks (e.g., photometric redshift estimation, the
construction of summary statistics) in order to be used for cosmological
constraints. The degree to which we can handle all of these technical
challenges will ultimately determine the accuracy of \mdet\ when used to analyze
imaging survey data.

The paper is laid out as follows... \mrb{do this}

\section{Analysis and Simulation Techniques}
\label{sec:sims}

In this section, we describe our object simulation and measurement techniques.
In all cases, we use the \galsim\ \citep{GALSIM2015} software package to
generate images, perform convolutions etc. We use the \texttt{SEP} \citep{sep}
Python wrapper of the Source Extractor software package \citep{Bertin96} for
source detection as needed. Finally, we make extensive use of \ngmix\ for object
measurement and the \mcal\ implementation.

\subsection{\textsc{Metacalibration}}

\mcal\ is a general technique that computes the linear response of image
measurements to an applied shear using only the observed image. Up to linear
order, we can write the response of an image measurement to an applied shear as

\begin{eqnarray}
\boldsymbol{e} & \approx & \left.\boldsymbol{e}\right|_{\gamma=0} +
                           \left.\frac{\partial \boldsymbol{e}}{\partial\boldsymbol\gamma}\right|_{\gamma=0} \boldsymbol\gamma +
                           O(\boldsymbol\gamma^2)\nonumber\\
               & \equiv  & \left.\boldsymbol{e}\right|_{\gamma=0} +
                           \boldsymbol{R} \boldsymbol\gamma +
                           O(\boldsymbol\gamma^2)
\end{eqnarray}
where $\boldsymbol{e}$ is the image measurement (e.g., object detection and shape
measurement), $\boldsymbol\gamma$ is the applied shear, and $\boldsymbol{R}$ is
the response (matrix) of the image measurement at zero applied shear,
$R_{ij}=\partial e_i /\partial \gamma_j$.

\mcal\ estimates this response via an numerical, finite-difference derivative
\begin{displaymath}
R_{ij} \approx \frac{e_i^{+} - e_i^{-}}{2\Delta\gamma_j}\ .
\end{displaymath}
where $R_{ij}$ is the estimated response of the object to shear
and $\Delta\gamma_j$ is a small shear, usually of order 0.01. The quantity
$e_i^{+/-}$ is the $i$-th shape component measured on an image sheared with
$\pm\Delta\gamma_j$. In detail, one must appropriately handle the PSF and
other observational effects when computing $R_{ij}$. Note that as the
estimated responses per-object are quite noisy, they are typically averaged
over many images/objects in order to estimate the response of a set of
images/objects to a shear.

In the notation above, we have denoted the object shape measurement as $\boldsymbol{e}$,
which implies both detection of objects {\it and} the shape measurements. In fact,
\mcal\ can measure the response of almost any image manipulation, including the
detection of objects as described in this work. From this perspective, \mdet\
is measuring the full response of an image (of possibly multiple objects) to
a constant shear.

\subsection{Multi-object Fitting Deblending}

Multi-object Fitting (MOF) deblending is a technique employed by the Dark
Energy Survey for accounting for blended objects when performing image
measurements \citep{DESY1cat}. It is representative of a set of techniques that
rely on fitting models to a set of preexisting detections. The model fit is
then used directly to form a flux measurement or indirectly by using it to
approximately remove the light of the neighboring objects in the image
before further processing. Typically, the models are a linear combination
of a bulge-like and disk-like component.

In this work, we use a MOF algorithm
that relies on \ngmix. It is an improved version of the MOF fitter from the
DES Y1 analysis which is both more stable and faster. It uses a linear combination
of a De Vaucouleurs' \citep{devauc1948} profile and exponential profile. The
profiles are constrained to be cocentric, coeliptical, and to have a fixed
fixed size ratio. The relative amplitude of the two profiles, the fraction of the
total flux in the De Vaucouleurs' profile, is a free parameter and is allowed
to vary outside of the range [0, 1]. Finally, to process a large number of
objects, we follow \citet{DESY1cat} and break them up into interacting groups.
These groups of objects are then simultaneously fit using a least-squares loss
function.

\subsection{Galaxy Pair Simulations}
\label{sec:sims:pairs}

In order to isolate the effects of detection easily, we employ a simulation setup
consisting of two simulated galaxies with a variable separation between them. By
varying the separation of the objects, we can turn on and off detection effects
in a controlled way.

The galaxies are each a combination of a bulge component, modeled as a De
Vaucouleurs' profile \citep{devauc1948} and disk component modeled as an
exponential. The fraction of light in the bulge was random and ranged uniformly
between 0.0 and 1.0. The disk ellipticity was drawn from the distribution
presented in \cite{ba14}, equation 24, with ellipticity variance set to 0.20,
with a random orientation. The bulge was given the same orientation as the disk
but with ellipticity set to the disk ellipticity times a random number drawn
uniformly between 0.0 and 0.5. The half-light radius of the disk
$r_{50}^{\mathrm{disk}}$ was set to a uniform random draw between 0.4 and 0.6
arcsec. The half-light radius of the bulge was a random draw between $0.4
r_{50}^{\mathrm{disk}}$ and $0.6 r_{50}^{\mathrm{disk}}$. The bulge was shifted
from the center of the disk within a radius 0.05$r_{50}^{\mathrm{disk}}$ and in
a random direction. The light of the disk was divided between a smooth component
and a set of simulated ``knots of star formation'', represented by point sources
placed randomly with the same exponential distribution as the disk.  Between 1
and 50 knots were placed, such that the fraction light in the knots ranged
between 0.4\% and 20\%. Finally, the total flux and noise were set such that the
signal-to-noise ratio ranged uniformly between 25 and 35.

We place two of these randomly generated galaxies in an image, with separation
ranging from 1.0 and 4.0 arcsec. The pair is placed such that the the line
between the pair had a uniform random orientation relative to the coordinate
axes. Each object was given an additional random dither within a pixel. The
galaxies were treated as transparent, such that the value in a pixel was equal
to the total sum from both galaxies plus noise. The pixel scale was set to 0.263
arcseconds, appropriate for a ground-based, DES-like survey. Example images are
shown in figure \ref{fig:pairs}

\begin{figure*}
    \includegraphics[width=\textwidth]{figures/bdk-comb.png}
    \caption{Example images of simulated galaxies used for the pair tests
    presented in section \ref{sec:sims:pairs}.  From left to right in the top row,
    the separations are 1.0, 1.5, 2.0. From left to right in the bottom row the
    separations are 3.0 and 4.0 arscec. The pixel scale is 0.263 arcsec.
    \label{fig:pairs}}
\end{figure*}

\subsection{Simulations with Representative Galaxy Density and Noise}
\label{sec:sims:realgals}

We use the \texttt{WeakLensingDeblending}\footnote{\url{https://github.com/LSSTDESC/WeakLensingDeblending}}
package in order to generate simulations with realistic galaxy densities and pixel noise for both the
DES and LSST surveys. We generated images in the r-, i-, and z-bands with an
effective depth that is roughly equivalent to full 5 and 10 year coadd image
for the DES and LSST respectively. We have neglected the effects of PSF
variation and use a constant PSF per-band with the typical
(expected) seeing for each survey ($\sim\!1$ arcsec and $\sim\!0.8$ arcsec resepctively).
For the DES image simulations, we have
fixed the exposure times for a single-epoch image to 90 seconds, so that
the full-depth coadd images have exposure times of 900 seconds. Example images
for each survey are shown in Figure~\ref{fig:simimages}.

\begin{figure*}
    \includegraphics[width=0.9\columnwidth]{figures/des_gri.jpg}
    \includegraphics[width=0.9\columnwidth]{figures/lsst_gri.jpg}
    \caption{
        Example images from the DES (left) and LSST (right) simulations. Each
        multicolor, $gri$-band image is approximately $\sim\!2.2$ arcmin on a side. The
        DES images have a pixel scale 0.263 arcsec and a PSF FWHM of $\sim\!1$ arcsec.
        The LSST images have a pixel scale of 0.2 arcsec and a PSF FWHM of $\sim\!0.8$
        arcsec.
        \label{fig:simimages}}
\end{figure*}


\section{Shear-dependent Detection Biases}

In this section, we employ a variety of specialized simulation setups to
elucidate the role of detection biases in \mcal\ shear measurements. We first
examine shear measurement on pairs of galaxies at various separations. We then
look at detection biases in DES- and LSST-like simulations with realistic
galaxy densities and pixel noise. We find in all cases that object detection
imparts a non-negligible shear measurement bias. As we will show in \S
\ref{sec:mdetpairs}, we can correct this bias by including detection in the
\mcal\ process, even if no explicit deblending is performed.

\subsection{Bias in Simulations of Galaxy Pairs}

We tested \mcal\ with MOF deblending using the galaxy pair simulation presented
in \S \ref{sec:sims:pairs}. Detection was performed using \sx, with settings
matching those used for DES year 5 survey reductions \ess{ref}.  We got similar
results using a simple local peak finder for detection.

The multiplicative bias $m$ is shown in figure \ref{fig:pairbias} as a function
of the pair distance. For a large separation of 4 arcsec, two objects are
detected in essentially all cases and no significant bias is seen.  For smaller
separations, the two objects overlap more significantly and the detection
becomes more ambiguous, with only one object detected in some cases.  The bias
grows as the separation decreases, with the maximum bias occurring at about 1.5
arcsec separation. At 1.5 arcsec separation the detection is most ambiguous,
with two objects detected in half the cases. For smaller separations the
objects overlap more and detection becomes less ambiguous, with one object
detected in more than half of the cases. At separations of 1 arcsec, the two
objects are detected as one essentially in every case, and again no significant
bias is detected.

\begin{figure}
    \includegraphics[width=\columnwidth]{figures/pairs-mc-bdkpair.pdf}

    \caption{Mean multiplicative shear bias measured for pairs of simulated
    galaxies (see \S \ref {sec:sims:pairs} for details) at various separations.  At
    each separation, a large number of trials was generated with random
    orientations of the pair.  At 4.0 arcsec separation, two objects were
    detected in all cases.  At 1.5 arcseconds two objects were detected in half
    the cases.  At 1.0 arcsec a single object was detected in all cases.  Red
    triangles represent standard \mcal\ with MOF deblending for modeling all
    detected objects.  Blue circles represent \mcal+MOF with detection included
    as part of the process.  Green pluses represent \mcal\ with detection
    included but without deblending, and using simple weighted moments without
    PSF correction as the shear estimator. Very large biases are seen for
    standard \mcal+MOF as detection becomes ambiguous, for example at 1.5
    arcsec separations.  When detection is included in the \mcal\ process the
    biases are greatly reduced.  The bias is reduced even in the case where no
    deblending was performed and no PSF correction or detailed object modeling
    were performed.  This indicates the large majority bias is due to
    shear-dependent detection, not light blending or details of the object
    modeling.
    \label{fig:pairbias}}

\end{figure}

The correspondence between detection ambiguity and shear bias is a hint that
the bias is caused by shear-dependent detection. Next, we perform simulations
of DES- and LSST-like surveys and show explicitly that by turning object
detection off, we can eliminate these detection biases.

\subsection{Bias in Simulations with Representative Galaxy Density and Noise}

We now explore the effects of detection in simulations that are representative
of real survey images. As a baseline, we note that \mcal\ with MOF
deblending and \sx\ detections results in catastrophic biases, $\sim-5\%$, for
a DES-like survey. Simulations of LSST-like surveys demonstrate even larger
biases. See Table~\ref{tab:shearmeas} for details. Note also that this
procedure closely matches that currently done with DES data.

In order to unpack the source of the bias in this case, we compare two different
\mcal\ shear measurements. The first \mcal\ shear measurement is done on a
catalog of the true source positions using a 1.2 arcsecond Gaussian weighted
moment shape measurement. The second employs \mcal\ with the same shape
measurement, but using \sx\ detections instead of the true object positions. We
find that while the first \mcal\ shear measurement is unbiased
($-0.0011\pm0.0012$), the second exhibits a bias of
$-0.058\pm0.001$.\footnote{The simulations with the true source positions are
computationally slow because they involve measurements on all sources, even ones
which are undetectable. Thus we were unable to decrease the errors on the
multiplicative bias below $\sim0.1\%$. However, for simulations with round,
exponential, high signal-to-noise objects only, we have found that \mcal\ with
the true source positions and a weighted moment is unbiased at high precision
($0.00033 +/- 0.00009$) in the presence of blending.} Note that this shape
measurement makes no corrections for object blending, but in the case where we
use the true object positions, it is still unbiased. Thus we have demonstrated
that given a set of true source locations, \mcal\ is not sensitive to
blending.\footnote{In practice, we have found that when using \mcal\ with true
detections and more complicated shape measurements, biases can reappear. Thus
whether or not a \mcal\ shape measurement is sensitive to blending depends
explicitly on the shape measurement technique used with the \mcal\ image
manipulations.}

This set of tests also demonstrates explicitly that source detection can cause
significant shear measurement biases even for techniques which are robust to
blending. The source detection biases probably originate from multiple causes.
One of those causes is undoubtedly related to merging or splitting objects in a
way which depends on the underlying shear, as illustrated with a toy example in
Figure~\ref{fig:toy} and the galaxy pair tests above. However, we have found
that even for extremely high signal-to-noise exponential objects placed on a
grid, using \mcal\ with the Gaussian weighted moments around object centers
determined by \sx\ still results in small biases, $\sim\!-0.2\%$. Thus at least
some of the biases have to do with the effect of the underlying shear on object
centering. Importantly, once we include object detection in the \mcal\ process,
we can correct for both of these effects, and any others, simultaneously.

% raw outputs
% DES mcal+MOF
% s2n: 10
%     # of sims: 49499
%     m       : -0.066781 +/- 0.009613
%     c       : -0.000055 +/- 0.000371
% s2n: 15
%     # of sims: 49496
%     m       : -0.050463 +/- 0.008567
%     c       : 0.000271 +/- 0.000416
% s2n: 20
%     # of sims: 49476
%     m       : -0.037302 +/- 0.007727
%     c       : 0.000201 +/- 0.000449

% LSST mcal+MOF
% s2n: 10
%     # of sims: 49014
%     m       : -0.082423 +/- 0.005180
%     c       : -0.000245 +/- 0.000192
% s2n: 15
%     # of sims: 49014
%     m       : -0.066753 +/- 0.004539
%     c       : -0.000173 +/- 0.000197
% s2n: 20
%     # of sims: 49014
%     m       : -0.062331 +/- 0.004279
%     c       : -0.000094 +/- 0.000210

% DES shear scene
% s2n: 10
%     # of sims: 9997419
%     m       : 0.000161 +/- 0.000857
%     c       : -0.000011 +/- 0.000034
% s2n: 15
%     # of sims: 9982020
%     m       : 0.000407 +/- 0.000786
%     c       : -0.000036 +/- 0.000039
% s2n: 20
%     # of sims: 9928501
%     m       : 0.000814 +/- 0.000754
%     c       : -0.000013 +/- 0.000044

% LSST shear scene
% s2n: 10
%     # of sims: 9996600
%     m       : 0.000472 +/- 0.000433
%     c       : -0.000024 +/- 0.000018
% s2n: 15
%     # of sims: 9996600
%     m       : -0.000015 +/- 0.000339
%     c       : -0.000003 +/- 0.000017
% s2n: 20
%     # of sims: 9996600
%     m       : 0.000190 +/- 0.000284
%     c       : -0.000012 +/- 0.000018

% DES no shear scene
% s2n: 10
%     # of sims: 9995117
%     m       : -0.003574 +/- 0.001228
%     c       : -0.000043 +/- 0.000037
% s2n: 15
%     # of sims: 9983339
%     m       : -0.000999 +/- 0.000997
%     c       : -0.000057 +/- 0.000040
% s2n: 20
%     # of sims: 9938170
%     m       : -0.002601 +/- 0.000948
%     c       : -0.000017 +/- 0.000044

% LSST no shear scene
% s2n: 10
%     # of sims: 9999000
%     m       : -0.004659 +/- 0.000559
%     c       : -0.000024 +/- 0.000018
% s2n: 15
%     # of sims: 9999000
%     m       : -0.003514 +/- 0.000430
%     c       : -0.000003 +/- 0.000018
% s2n: 20
%     # of sims: 9999000
%     m       : -0.002873 +/- 0.000367
%     c       : -0.000022 +/- 0.000019


\begin{table*}
  \centering
  \caption{
    Multiplicative biases in weak lensing simulations for various shear
    measurement techniques. In all cases, the simulations use realistic
    galaxy shapes, galaxy sizes and noise for the given survey. For measurements using standard \mcal\ with
    MOF deblending, a cut of $T/T_{PSF} > 0.5$ was also applied. Measurements with
    \mdet\ and moments used a size cut of $T/T_{PSF} > 1.2$. In the case of \mdet\ with moments,
    no deblending corrections are applied and the moments are a simple weighted moment
    with no PSF correction.}
  \label{tab:shearmeas}

  \begin{tabular}{|l|l|l|c|c|}
    \hline
    Simulation & Method & Full Scene Sheared? & \snr\ Cut & m \\
    \hline

    \hline
    \multicolumn{5}{c}{metacal+MOF - full scene sheared}\\
    \hline
    DES   & metacal+MOF & yes & \snr$ > 10$ & $-0.067 \pm 0.010$  \\
    DES   & metacal+MOF & yes & \snr$ > 15$ & $-0.050 \pm 0.009$  \\
    DES   & metacal+MOF & yes & \snr$ > 20$ & $-0.037 \pm 0.008$  \\
    \hline
    LSST  & metacal+MOF & yes & \snr$ > 10$ & $-0.082 \pm 0.005$  \\
    LSST  & metacal+MOF & yes & \snr$ > 15$ & $-0.067 \pm 0.005$  \\
    LSST  & metacal+MOF & yes & \snr$ > 20$ & $-0.062 \pm 0.004$  \\
    \hline

    \hline
    \multicolumn{5}{c}{metadetect+moments - full scene sheared}\\
    \hline
    DES   & metadetect+moments & yes & \snr$ > 10$ & $+0.00016 \pm 0.00086$  \\
    DES   & metadetect+moments & yes & \snr$ > 15$ & $+0.00041 \pm 0.00079$  \\
    DES   & metadetect+moments & yes & \snr$ > 20$ & $+0.00081 \pm 0.00075$  \\
    \hline
    LSST  & metadetect+moments & yes & \snr$ > 10$ & $+0.00047 \pm 0.00043$  \\
    LSST  & metadetect+moments & yes & \snr$ > 15$ & $-0.00002 \pm 0.00034$  \\
    LSST  & metadetect+moments & yes & \snr$ > 20$ & $+0.00019 \pm 0.00028$  \\
    \hline

    \hline
    \multicolumn{5}{c}{metadetect+moments - individual objects sheared}\\
    \hline
    DES   & metadetect+moments & no & \snr$ > 10$ & $-0.0036 \pm 0.0012$  \\
    DES   & metadetect+moments & no & \snr$ > 15$ & $-0.0010 \pm 0.0010$  \\
    DES   & metadetect+moments & no & \snr$ > 20$ & $-0.0026 \pm 0.0009$  \\
    \hline
    LSST  & metadetect+moments & no & \snr$ > 10$ & $-0.0047 \pm 0.0006$  \\
    LSST  & metadetect+moments & no & \snr$ > 15$ & $-0.0035 \pm 0.0004$  \\
    LSST  & metadetect+moments & no & \snr$ > 20$ & $-0.0029 \pm 0.0004$  \\
    \hline
  \end{tabular}

\end{table*}


\section{Mitigating Shear-dependent Detection Biases with \textsc{METACALIBRATION}}

In the previous section, we demonstrated that source detection is a significant
source of bias in shear measurements with \mcal. Here we test a straight-forward
fix for this bias, namely including source detection in the \mcal\ process. We
have implemented this fix as follows. We take relatively large images (typically
1 arcminute on a side) and apply \mcal\ directly to the full image using the
PSF at the center of the image. For each of the five \mcal\ images, we apply
\sx\ to detect sources. We then measure shears in postage stamps around each
detection in each image using an non-PSF corrected, Gaussian-weighted moment.
These five catalogs are then combined into a single estimate for the shear in
the patch by computing the average shear response for the image
\begin{eqnarray}
\langle \boldsymbol\gamma \rangle &\approx& \langle \boldsymbol{R}\rangle^{-1}\langle\boldsymbol{e}\rangle\\
\langle R_{ij}\rangle &=& \frac{\langle e_i^{+}\rangle - \langle e_i^{-}\rangle}{2\Delta\gamma_j}
\end{eqnarray}
where the equation for the response above is for a single shear component. We term this
process \mdet.

Note that the averages above are over the {\it catalogs} from running source detection
and shape measurement on each sheared image. We make no attempt to match objects between the different
sheared images. In fact, doing so would potentially bring back in the shear
detection biases since using only matched objects would impart a shear dependent
selection into the matched catalogs. This last fact has implications for
analyses using \mdet\ in the analysis of surveys where reference or "gold"
samples of galaxies are typically constructed. See more discussion in the
conclusions below.

\subsection{Results for Simulated Galaxy Pairs}
\label{sec:mdetpairs}

In figure \ref{fig:pairbias} we show results for the pairs of galaxies, including
detection in the \mcal\ process. The blue filled circles represent the case
where deblending is performed using MOF. The green plus signs represent the
case where no deblending was performed. For the undeblended case we further
simplified the process: simple weighted moments were taken at the position
determined by \sx\ using a fixed weight function with full-width at half
maximum 1.2 arcsec, without any correction for the PSF.

In both cases the bias is greatly reduced, with significant bias seen only at
the special separation of 1.5 arcsec, where the two objects are detected as one
object in half of the cases. This demonstrates that the bias we see is not
primarily due to the process of deblending itself, but rather shear-dependent
detection effects. The remaining biases at 1.5 arcsec tend to be different
sign for the deblended and non-deblended cases, which shows there is a
qualitative difference in how the two measurements respond to the shear. As we
will show below, we find no significant bias for more realistic DES and
LSST-like images where the typical separation of galaxies is not at a special
location of maximum detection ambiguity.

\subsection{Results for Simulations with Representative Galaxy Density and Noise}
\label{sec:res:constpsf}

We show results for DES-like and LSST-like surveys in Table~\ref{tab:shearmeas}.
We have assumed a constant PSF and constant shear for these simulations. We find
that in all cases our \mdet\ shear measurements are unbiased up to
second-order shear effects. This conclusion holds despite the extensive blending
of the object images and the large source detection effects we decoumented
above. They also meet or exceed the requirements for analyzing an LSST-like survey
\citep[e.g.,][]{huterer2006}.

Finally, we address a key physical assumption made by \mdet, namely that of constant
shear. In a constant shear image, the space between objects is sheared in the
same way the objects are. This shearing of the full scene matches the \mdet\
shearing steps exactly and thus allows us to calibrate the mean shear in the
image to high precision. A realistic image with a mix of objects under different
shears at different redshifts can be thought of as a sum of series of constant
shear images with the full scene sheared independently for each image in the sum.
This image would not exactly match the \mdet\ process and thus we might expect
some residual bias in the shear calibration.

In order to obtain an estimate and bound on this effect, we make a simple
modification to our simulations. When building them, instead of shearing the
full scene to impart the true shear to the image, we shear each object
individually and then add it to the image. This modification leaves the space
between objects maximally unsheared. Results for these simulations are in the
bottom rows of Table~\ref{tab:shearmeas}. We find small residual biases in this
case, of order $\sim-0.3\%$. Further, we expect these biases to depend on
the density of blended objects in the survey, with larger residual biases
for LSST-like surveys than DES-like surveys.

\section{Moving Beyond Simple Simulations: Handling PSF Variation}

In order to apply \mdet\ to a real, multi-band, multi-epoch survey like the
DES or LSST, we will need to be able to handle realistic levels of PSF variation,
missing data, and non-trivial WCS transformations. We will address these issues
more fully in future work currently in preparation. However, here we will address
the issue of PSF variation, which is technically more challenging since \mdet\
requires us to deconvolve the PSF over relatively large regions of sky. Here, we
will show that the image coadding process used in a realistic multiepoch survey
naturally reduces the PSF variation enough to render this effect negligible.

For these simulations, we use the same DES-like and LSST-like simulations as above,
but create the PSF model by stacking some number of variable PSF models. \mdet\ is then
performed using the stacked PSF model at the center of the image. In order
to place an upper bound on this effect, we have created a variable PSF model that
has significantly more variation than we expect in real data. See Appendix~\ref{app:pspsf}
for details. With only stacking one of the PSF models from Appendix~\ref{app:pspsf},
we find an multiplicative bias of $XYZ$. However, once we stack five of these models,
which is a minimum amount of coadding expected for both the DES and LSST, we find
an multiplicative bias of only $XYZ$. Thus we expect that in a realistic survey
scenario, PSF variation is not a fundamental limitation for \mdet.


\section{Summary}

In this work, we have explored how \mcal\ weak lensing measurements perform in
realistic scenarios where a large fraction of the objects overlap and the detection
of objects can be ambiguous. These conditions will characterize all future weak
lensing surveys and so accurate performance in this regime is critical to getting
the most scientific gains from the surveys. We find that \mcal\ used with MOF-like
object deblending techniques has severe, many percent biases that get worse with
the degree of object blending. We then demonstrated that by including the detection
of objects in the \mcal\ process, we can eliminate these biases at high precision
even for an LSST-like survey. This technique, which we call \mdet, appears to be
extremely promising for future surveys.

Future work will be needed in two directions in order to fully realize the
potential of \mdet. First, specific procedures for the multi-epoch image coadding,
the handling of missing data, and the handling of the PSF variation will be needed to
apply \mdet\ to real survey data. We have work in preparation already to develop
these procedures for the DES and LSST. Further, we demonstrated above that PSF
variation naturally averages down during the image coaddition process so that it
can safely be ignored. Second, we will need procedures to process the \mdet\ outputs,
which are five catalogs for the full survey, into the various summary statistics
(e.g., shear correlation functions) used to constrain cosmological models. In particular,
care must be taken to not match these catalogs to one another or some external catalog
because this matching process could reintroduce the shear-dependent object detection biases.
The solutions to these issues will be the limiting factor in \mdet.

\bibliographystyle{mnras}
\bibliography{references}

\appendix

\section{Fast Approximate Variable PSF Models}\label{app:pspsf}

In this work we use a fast, approximate variable PSF model. This model eases the
computational requirements for the simulations while also retaining the
essential features of realistic PSF variation. In this appendix, we present
the model and verify its statistical properties against more realistic PSF models.

We start with the results of \citet{heymans2012}. They fit the \vonkarman model
of atmospheric turbulence

\begin{displaymath}
  P(\ell) \propto \left(\ell^{2} + \frac{1}{\theta_{0}^2}\right)^{-11/6}
\end{displaymath}
to images with high stellar density. Here $\theta_{0}$ is the outer scale of
turbulence. \citep{heymans2012} find that $\theta_{0}\approx3$ arcmin.
We further add an additional Gaussian truncation of the power

\begin{displaymath}
  P_{trunc}(\ell) \propto P(\ell)\exp\left(-\ell^2r^{2}\right)
\end{displaymath}
at a scale of $r=1$ arcsec in order to reduce the level of resulting
PSF variation. Below we show that even with this modification, our models
still have more power than a realistic model for a survey, making them useful
for providing upper limits on the effects of PSF variation.

Using this model, we seed equal amounts of E- and B-mode power on a grid of
$128\times128$ cells using random phases. Each cell of the grid is one
arcsec in size. We normalize the overall shape variance to $0.05^2$. We then use
the $g1$ and $g2$ components of this model to set the shape of the PSF at each
location. Note that we also bound the total ellipticity to at most 0.5.
We model the PSF profile as a Moffat with shape parameter $\beta=2.5$.
The size of the Moffat profile is set to be proportional to $\mu^{-3/4}$,
where $\mu$ is the magnification computed from the power spectra realization. The
proportionality constant is drawn randomly from a log-normal model with
scatter 0.1 arcmin and a central value set so the final PSF size. This mimics
the typical seeing conditions of a given survey.

We show an example PSF for a DES-like survey in Figure~\ref{fig:pspsf}.  Over a
1 square arcminute patch, our approximate models generate PSF shape and size
variation that are $\gtrsim10\times$ that seen in real 90 second exposures
with DECam \citep{DESY1shear}, or the expected variation in a 15 second exposure with
LSST \citep{jee2011} over similar scales. Figure~\ref{fig:psxi} shows the $\xi_{\pm}$ shear correlation
functions averaged over 100 realizations of our models. For comparison, we
expect at most shear correlation function amplitudes of $\sim10^{-4}$ for LSST
\citep{jee2011} and for DESCam 90 second exposures. The DECam models were generated
using the methods of \citep{jee2011} but for DECam-like environmental conditions. For
the optical contributions to the PSF, we use a set of randomly drawn
aberrations (similar to GREAT3 \citep{great3}), but with values more typical of
DECam
observations\footnote{\url{https://github.com/GalSim-developers/GalSim/blob/releases/2.1/examples/great3/cgc.yaml}}.
Finally, we note that simulations of \mdet\ with this PSF model show
$\approx-0.8\%$ multiplicative biases for a DES-like survey at full depth but
without simulating the reduction of PSF variation due to stacking.

\begin{figure*}
  \includegraphics[width=\textwidth]{figures/pspsf.pdf}
  \caption{
    Variable PSF model statistics for a DECam-like exposure. The top-left
    panel shows the variation in the FWHM in arcseconds. The top-right panel
    shows a visualization of the PSF shape variation. The bottom-left panel shows
    the variation in the $1$-component of the PSF shape. The bottom-right panel
    shows the variation in the $2$-component of the PSF shape. The variation in
    this model is $\gtrsim10\times$ larger than the typical PSF variation for
    either DECam or expected LSST observations.
    \label{fig:pspsf}}
\end{figure*}

\begin{figure}
  \includegraphics[width=\columnwidth]{figures/psxi.pdf}
  \caption{
    Variable PSF model shear correlation functions for a DECam-like exposure. LSST
    is expected to have shear correlation function magnitudes around
    $\sim\!10^{-4}$ \citep{jee2011}.
    \label{fig:psxi}}
\end{figure}


\bsp
\label{lastpage}
\end{document}
