\documentclass[fleqn,useAMS,usenatbib]{mnras}
%=====================================================================
% CUSTOM: PACKAGES, MACROS & SETTINGS
%=====================================================================
% packages for figures
\usepackage{graphicx,todonotes}

% packages for symbols
\usepackage{latexsym,amssymb}

% AMS-LaTeX package for e.g. subequations
\usepackage{amsmath,morefloats}
\usepackage{natbib,graphicx,amsmath,subfigure,color,xcolor,hyperref}

\topmargin-1cm

\graphicspath{{figures/}}

\newcommand\notedo[1]{\todo[color=yellow, inline, size=\small]{To do:#1}}
\newcommand\notewrite[1]{\todo[color=orange, inline, size=\small]{To write: #1}}
\newcommand\noteask[1]{\todo[color=cyan, inline, size=\small]{To ask: #1}}
\newcommand\notecontrib[1]{\todo[color=green, inline, size=\small]{Contributors: #1}}
\newcommand\esstodo[1]{\todo[color=yellow, inline, size=\small]{ESS: #1}}
\newcommand{\ess}[1]{\textcolor{red}{[ESS: \bf #1]}}
\newcommand{\mrb}[1]{\textcolor{purple}{[MRB: \bf #1]}}


\newcommand{\vecg}{\mbox{\boldmath $g$}}
\newcommand{\vece}{\mbox{\boldmath $e$}}
\newcommand{\veck}{\mbox{\boldmath $k$}}
\newcommand{\vecQ}{\mbox{\boldmath $Q$}}
\newcommand{\vecF}{\mbox{\boldmath $F$}}
\newcommand{\vecD}{\mbox{\boldmath $D$}}
\newcommand{\matR}{\mbox{$\bf R$}}
\newcommand{\matC}{\mbox{$\bf C$}}
\newcommand{\bnab}{\boldsymbol{\nabla}}
\newcommand{\bnabg}{\boldsymbol{\nabla_g}}
\newcommand{\galsim}{\texttt{GALSIM}}
\newcommand{\ngmix}{\texttt{ngmix}}
\newcommand{\nnsim}{\texttt{nsim}}
\newcommand{\snr}{$S/N$}
\newcommand{\sn}{$S/N$}
\newcommand{\coadd}{{\rm coadd}}
\newcommand{\desreq}{$4\times 10^{-3}$}
\newcommand{\lsstreq}{$2\times 10^{-3}$}

\newcommand{\mcal}{\textsc{metacalibration}}
\newcommand{\mdet}{\textsc{metadetection}}
\newcommand{\Mcalshort}{\textsc{metacal}}
\newcommand{\Mcal}{\textsc{Metacalibration}}
\newcommand{\Mdet}{\textsc{Metadetection}}
\newcommand{\vest}{\mbox{\boldmath $e$}}
\newcommand{\est}{e}
\newcommand{\mcalR}{\mbox{\boldmath $R$}}
\newcommand{\mcalRS}{\mbox{\boldmath $R_S$}}
\newcommand{\gest}{\mbox{\boldmath $\hat \gamma$}}
\newcommand{\vecgam}{\mbox{\boldmath $\gamma$}}

\newcommand{\sx}{\textsc{SExtractor}}

\newcommand{\bfd}{\textsc{BFD}}

\newcommand{\vonkarman}{{von K\'arm\'an}~}

\title[\Mdet]{\Mdet: Mitigating Shear-dependent Detection Biases with \Mcal}

\author[Sheldon et~al.]{Erin Sheldon$^1$, Matthew R. Becker$^2$,
Niall MacCrann$^{3,4}$, Michael Jarvis$^5$
  \\$^1$Brookhaven National Laboratory, Bldg. 510, Upton, NY 11973, USA
  \\$^2$High Energy Physics Division, Argonne National Laboratory, Lemont, IL 60439, USA
  \\$^3$Center for Cosmology and Astro-Particle Physics, The Ohio State University, Columbus, OH 43210, USA
  \\$^4$Department of Physics, The Ohio State University, Columbus, OH 43210, USA
  \\$^5$Department of Physics and Astronomy, University of Pennsylvania, Philadelphia, PA 19104, USA
}

\begin{document}
\date{Draft \today}
\maketitle

\begin{abstract}

\Mcal\ is a new technique for measuring weak gravitational lensing shear that
is unbiased for isolated galaxy images.  In this work we test the \mcal\ method
with overlapping, or ``blended'' galaxy images.  Using standard \mcal\ we find
a significant shear calibration bias when objects overlap. This bias is a few
percent for galaxy densities relevant for current surveys, and increases to
nearly ten percent for future surveys.  This bias is due to shear-dependent
detection rather than blending itself; if detection is shear independent, no
de-blending of images is needed, in principle.  We demonstrate that
shear-dependent detection biases are accurately removed when including
detection in the \mcal\ process, a technique we call \mdet.

\end{abstract}

\section{Introduction}


Recently introduced methods for the estimation of weak gravitational lensing
shear promise to provide calibration at the 0.1\% level, adequate for the
requirements of future weak lensing surveys.  Two methods that have
demonstrated sufficient accuracy without the use of calibration from
simulations, and without any compromise of precision, are the \bfd\ method
\citep{BernBFD2016} and the \mcal\ method \citep{HuffMcal2017,SheldonMcal2017}.
The tests of these methods, while stringent, did not include an important
aspect of the real universe: the images of objects overlap on the sky. In this
work we test \mcal\ in the presence of blending, and demonstrate that there is
a bias associated with shear-dependent detection. We introduce an improvement
to the method that naturally accounts for this shear-dependent detection bias.

A method such as \mcal\ can in principle calibrate any measurement biases, even
those associated with blending \citep[e.g.,][]{DawsonBlending2016}. More problematic for
shear calibration is the dependence of the object detection on shear. In the
weak shear regime, the lensing mapping is simple shear transformation that
preserves surface brightness \citep{SchneiderBook92}. For such a simple mapping, detection need not
be shear dependent. For example, a detection algorithm that simply identifies
connected regions above a threshold as a single object will not in principle be
shear dependent. However, in real observations the image is smeared by a
point-spread-function (PSF) due to the atmosphere, telescope optics or detector.
In this case the overlap of objects does depend on shear because the PSF
convolution occurs after the shear mapping. In this case the simple threshold
detection method mentioned above will manifest a shear-dependent detection bias.
This effect is demonstrated in figure \ref{fig:toy}.


\begin{figure*}
  %\includegraphics[width=\textwidth]{figures/toy.pdf}
  \includegraphics[width=\textwidth]{figures/toy.png}

  \caption{ Toy example of shear-dependent detection in the presence of a
  PSF.  In panel (a) two objects are present, convolved by a PSF with no
  shear.  Contours represent constant brightness.  In panel (b) the objects
  are sheared by $\gamma = (0.0, 0.1)$ {\em after} the PSF convolution.
  Contours are the same as panel (a).  In this case the inner contours for
  the two objects overlap before and after application of the shear.  This is
  a general property of the shear transformation in the weak regime. In panel
  (c) the shear is applied {\em before} the PSF convolution, which mimics real
  sky images. In this case the inner contours do not overlap after shearing.
  For case (c) a detection algorithm that identified connected regions above
  a threshold as a single object would manifest a shear-dependent detection
  bias.  \label{fig:toy} }

\end{figure*}

Common detection schemes in use today, such at those in Source Extractor
\citep{Bertin96} and the HST/LSST pipelines \citep{BoschHSC2018,BoschLSST2018}
are based on thresholding, similar to the simple approach described above but
differing in complexity and efficiency. Thus we would expect detections
produced by those codes to also manifest shear dependence. An open question,
which we will not address in this work, is whether or not one can derive a
detection algorithm which is independent of the shear applied to the image
in the presence of PSF and detector distortions. Such an algorithm would
in principle eliminate the shear-dependent detection biases explored in this
work.

Current implementations of \mcal\ \citep[e.g.,][]{HuffMcal2017,SheldonMcal2017},
when used with the common detection schemes discussed above, are expected to exhibit shear-dependent
detection biases. These implementations of \mcal\ work by applying artificial
shears to small postage stamp images for objects found in a {\em preexisting, independent}
detection step. This detection step has the shear-dependent detection bias in it already,
and thus any shear applied when running \mcal\ does not properly measure the full shape response
to shear. Note that corrections for selection effects were derived in
\cite{SheldonMcal2017}, but those do not work near the detection threshold since objects
needed for the corrections may not have been found at all. As we will
demonstrate, detection effects can be incorporated naturally by shearing larger
images, rather than small postage stamps, and re-running the detection algorithm on
each of the sheared images. In this process, shear measurement errors,
selection effects and detection effects are all accounted for. We call this technique
\mdet.

There are a number of technical challenges associated with \mdet, only some of which
we will address in this work. First, larger images can have non-trivial
PSF and world coordinate system (WCS) variation. Usually these observational details
are approximately constant over small post-stamps and so can be dealt with easily.
Second, larger regions of images can have non-trivial masking patterns due to
stellar diffraction spikes, cosmic rays, etc. In the implementation presented in
this work, we use fast Fourier transforms (FFTs) to handle convolutions, so that
this data needs to be interpolated in some way. Finally, \mdet\ involves shearing
a large patch of sky, including shearing the space between objects. This operation
is qualitatively different from shearing each object about its center (what \mcal\ does)
or from the shearing applied by a cosmological density field (which is a mixture of
the two). The degree to which we can handle these technical challenges will ultimately
determine the accuracy of \mdet\ when used to analyze imaging survey data.


The paper is laid out as follows.... blah blah.

\section{Analysis and Simulation Techniques}
\label{sec:sims}

In this section, we describe our object simulation and measurement techniques. In
all cases, we use the \galsim\ \citep{GALSIM2015} software package
to generate images, perform convolutions etc. We use the \texttt{SEP} \citep{sep}
Python wrapper of the Source Extractor software package \citep{Bertin96} for
source detection as needed. Finally, we make extensive use of \ngmix\
for object measurement and the \mcal\ implementation.

\subsection{\textsc{Metacalibration}}

\mcal\ is a general technique that computes the linear response of image measurements
to an applied shear using only the observed image. Up to linear order, we can
write the response of an image measurement to an applied shear as

\begin{eqnarray}
f(I, \gamma) & \approx & f(I, \gamma=0) + \left.\frac{\partial f}{\partial\gamma}\right|_{\gamma=0} \gamma + O(\gamma^2)\nonumber\\
&\equiv& f(I, \gamma=0) + R \gamma + O(\gamma^2)
\end{eqnarray}
where $f(I,\gamma)$ is the image measurement (e.g., object detection and shape
measurement), $I$ is the image, $\gamma$ is the applied shear, and $R$ is the response of the
image measurement at zero applied shear.

\mcal\ estimates this response via a
numerical, finite-difference derivative
\begin{displaymath}
  \hat{R} \approx \frac{f(I, +\epsilon) - f(I, -\epsilon)}{2\epsilon}\ .
\end{displaymath}
where $\hat{R}$ is the estimated response of the object to shear
and $\epsilon$ is a small shear, usually of order 0.01. In detail, one must
appropriately handle the PSF and other observational effects when computing
$\hat R$. Note that as the estimated responses per-object are quite noisy,
they are typically averaged over many images/objects in order to estimate
the response of a set of images/objects to a shear.

In the notation above, we have denoted the object shape measurement as
$f(I, \gamma)$ representing the measurement performed on an image at some applied
shear. This notation reflects the fact that \mcal\ can in fact measure the response
of almost any image manipulation, including the detection of objects as described
in this work. From this perspective, \mdet\ is measuring the full response of
an image (of possibly multiple objects) to a constant shear.

\subsection{Multi-object Fitting Deblending}

Multi-object Fitting (MOF) deblending is a technique employed by the Dark
Energy Survey for accounting for blended objects when performing image
measurements \citep{DESY1cat}. It is representative of a set of techniques that
rely on fitting models to a set of preexisting detections. The model fit is
then used directly to form a flux measurement or indirectly by using it to
approximately remove the light of the neighboring objects in the image
before further processing. Typically, the models are a linear combination
of a bulge-like and disk-like component.

In this work, we use a MOF algorithm
that relies on \ngmix. It is an improved version of the MOF fitter from the
DES Y1 analysis which is both more stable and faster. It uses a linear combination
of a De Vaucouleurs' \citep{devauc1948} profile and exponential profile. The
profiles are constrained to be cocentric, coeliptical, and to have a fixed
fixed size ratio. The relative amplitude of the two profiles, the fraction of the
total flux in the De Vaucouleurs' profile, is a free parameter and is allowed
to vary outside of the range [0, 1]. Finally, to process a large number of
objects, we follow \citet{DESY1cat} and break them up into interacting groups.
These groups of objects are then simultaneously fit using a least-squares loss
function.

\subsection{Galaxy Pair Simulations}
\label{sec:sims:pairs}

In order to isolate the effects of detecton easily, we employ a similation setup
consisting of two simulated galaxies with a variable separation between them. By
varying the separation of the objects, we can turn on and off detection effects in
a controlled way.

The galaxies are each a combination of a bulge component, modeled as a
De Vaucouleurs' profile \citep{devauc1948} and disk component modeled as
an exponential. The fraction of light in the bulge was random and ranged
uniformly between 0.0 and 1.0. The disk ellipticity was drawn from the
distribution presented in \cite{ba14}, equation 24, with ellipticity variance
set to 0.20, with a random orientation. The bulge was given the same orientation
as the disk but with ellipticity set to the disk ellipticity times a random
number drawn uniformly between 0.0 and 0.5. The half-light radius of the disk
$r_{50}^{\mathrm{disk}}$ was set to a uniform random draw between 0.4 and 0.6
arcsec. The half-light radius of the bulge was a random draw between $0.4 r_{50}^{\mathrm{disk}}$
and $0.6 r_{50}^{\mathrm{disk}}$. The bulge was shifted from the center of the
disk within a radius 0.05$r_{50}^{\mathrm{disk}}$ and in a random direction.
The light of the disk was divided between a smooth component and a set of
simulated ``knots of star formation'', represented by point sources placed
randomly with the same exponential distribution as the disk.  Between 1 and 50
knots were placed, such that the fraction light in the knots ranged between
0.4\% and 20\%. Finally, the total flux and noise were set such that the
signal-to-noise ratio ranged uniformly between 25 and 35.

We place two of these randomly generated galaxies in an image, with
separation ranging from 1.0 and 4.0 arcsec. The pair is placed such that the
the line between the pair had a uniform random orientation relative to the
coordinate axes. Each object was given an additional random dither within a
pixel. The galaxies were treated as transparent, such that the value in a
pixel was equal to the total sum from both galaxies plus noise. The pixel scale
was set to 0.263 arcseconds, appropriate for a ground-based, DES-like survey.
Example images are shown in figure \ref{fig:pairs}

\begin{figure*}
    \includegraphics[width=\textwidth]{figures/bdk-comb.png}
    \caption{Example images of simulated galaxies used for the pair tests
    presented in section \ref{sec:sims:pairs}.  From left to right in the top row,
    the separations are 1.0, 1.5, 2.0. From left to right in the bottom row the
    separations are 3.0 and 4.0 arscec. The pixel scale is 0.263 arcsec.
    \label{fig:pairs}}
\end{figure*}

\subsection{Simulations with Representative Galaxy Density and Noise}
\label{sec:sims:realgals}

\begin{figure*}
    \includegraphics[width=0.9\columnwidth]{figures/des-rgb-000000-crop.png}
    \includegraphics[width=0.9\columnwidth]{figures/lsst-rgb-000003-crop.png}

    \caption{Example images from the DES (left) and LSST (right) simulations.
    \label{fig:simimages}}
\end{figure*}


\section{Shear-dependent Detection Biases}

\subsection{Bias in Simulations of Galaxy Pairs}

We tested \mcal\ with MOF deblending using the galaxy pair simulation presented
in \S \ref{sec:sims:pairs}.  Detection was performed using \sx, with settings
matching those used for DES year 5 survey reductions \ess{ref}.  We got similar
results using a simple local peak finder for detection.

The multiplicative bias $m$ is shown in figure \ref{fig:pairbias} as a function
of the pair distance.  For a large separation of 4 arcsec, two objects are
detected in essentially all cases and no significant bias is seen.  For smaller
separations, the two objects overlap more significantly and the detection
becomes more ambiguous, with only one object detected in some cases.  The bias
grows as the separation decreases, with the maximum bias occurring at about 1.5
arcsec separation.  At 1.5 arcsec separation the detection is most ambiguous,
with two objects detected in half the cases.  For smaller separations the
objects overlap more and detection becomes less ambiguous, with one object
detected in more than half of the cases.  At separations of 1 arcsec, the two
objects are detected as one essentially in every case, and again no significant
bias is detected.

\begin{figure}
    \includegraphics[width=\columnwidth]{figures/pairs-mc-bdkpair.pdf}

    \caption{Mean multiplicative shear bias measured for pairs of simulated
    galaxies (see \S \ref {sec:sims:pairs} for details) at various separations.  At
    each separation, a large number of trials was generated with random
    orientations of the pair.  At 4.0 arcsec separation, two objects were
    detected in all cases.  At 1.5 arcseconds two objects were detected in half
    the cases.  At 1.0 arcsec a single object was detected in all cases.  Red
    triangles represent standard \mcal\ with MOF deblending for modeling all
    detected objects.  Blue circles represent \mcal+MOF with detection included
    as part of the process.  Green pluses represent \mcal\ with detection
    included but without deblending, and using simple weighted moments without
    PSF correction as the shear estimator. Very large biases are seen for
    standard \mcal+MOF as detection becomes ambiguous, for example at 1.5
    arcsec separations.  When detection is included in the \mcal\ process the
    biases are greatly reduced.  The bias is reduced even in the case where no
    deblending was performed and no PSF correction or detailed object modeling
    were performed.  This indicates the large majority bias is due to
    shear-dependent detection, not light blending or details of the object
    modeling.
    \label{fig:pairbias}}

\end{figure}

The correspondence between detection ambiguity and shear bias is a hint that
the bias is caused by shear-dependent detection.  As we will show in \S
\ref{sec:mdetpairs}, we can correct this bias by including detection in the
\mcal\ process, even if no explicit deblending is performed.

\subsection{Bias in Simulations with Representative Galaxy Density and Noise}

Erin's tests with WeakLensingDeblending and Matt's tests

% raw outputs
% DES mcal+MOF
% s2n: 10
%     # of sims: 49499
%     m       : -0.066781 +/- 0.009613
%     c       : -0.000055 +/- 0.000371
% s2n: 15
%     # of sims: 49496
%     m       : -0.050463 +/- 0.008567
%     c       : 0.000271 +/- 0.000416
% s2n: 20
%     # of sims: 49476
%     m       : -0.037302 +/- 0.007727
%     c       : 0.000201 +/- 0.000449

% LSST mcal+MOF
% s2n: 10
%     # of sims: 49014
%     m       : -0.082423 +/- 0.005180
%     c       : -0.000245 +/- 0.000192
% s2n: 15
%     # of sims: 49014
%     m       : -0.066753 +/- 0.004539
%     c       : -0.000173 +/- 0.000197
% s2n: 20
%     # of sims: 49014
%     m       : -0.062331 +/- 0.004279
%     c       : -0.000094 +/- 0.000210

% DES shear scene
% s2n: 10
%     # of sims: 9997419
%     m       : 0.000161 +/- 0.000857
%     c       : -0.000011 +/- 0.000034
% s2n: 15
%     # of sims: 9982020
%     m       : 0.000407 +/- 0.000786
%     c       : -0.000036 +/- 0.000039
% s2n: 20
%     # of sims: 9928501
%     m       : 0.000814 +/- 0.000754
%     c       : -0.000013 +/- 0.000044

% LSST shear scene
% s2n: 10
%     # of sims: 9996600
%     m       : 0.000472 +/- 0.000433
%     c       : -0.000024 +/- 0.000018
% s2n: 15
%     # of sims: 9996600
%     m       : -0.000015 +/- 0.000339
%     c       : -0.000003 +/- 0.000017
% s2n: 20
%     # of sims: 9996600
%     m       : 0.000190 +/- 0.000284
%     c       : -0.000012 +/- 0.000018

% DES no shear scene
% s2n: 10
%     # of sims: 9995117
%     m       : -0.003574 +/- 0.001228
%     c       : -0.000043 +/- 0.000037
% s2n: 15
%     # of sims: 9983339
%     m       : -0.000999 +/- 0.000997
%     c       : -0.000057 +/- 0.000040
% s2n: 20
%     # of sims: 9938170
%     m       : -0.002601 +/- 0.000948
%     c       : -0.000017 +/- 0.000044


\begin{table*}
  \centering
  \caption{
    Multiplicative biases in weak lensing simulations for various shear
    measurement techniques. In all cases, the simulations use realistic
    galaxy shapes, galaxy sizes and noise for the given survey. For measurements using standard \mcal\ with
    MOF deblending, a cut of $T/T_{PSF} > 0.5$ was also applied. Measurements with
    \mdet\ and moments used a size cut of $T/T_{PSF} > 1.2$. In the case of \mdet\ with moments,
    no deblending corrections are applied and the moments are a simple weighted moment
    with no PSF correction.}
  \label{tab:shearmeas}

  \begin{tabular}{|l|l|l|l|c|c|}
    \hline
    Simulation & Method & Full Scene Sheared? & PSF & \snr\ Cut & m \\
    \hline
    \hline
    \multicolumn{6}{c}{metacal+MOF - full scene sheared}\\
    \hline
    \hline
    DES   & metacal+MOF & yes & constant & \snr$ > 10$ & $-0.067 \pm 0.010$  \\
    DES   & metacal+MOF & yes & constant & \snr$ > 15$ & $-0.050 \pm 0.009$  \\
    DES   & metacal+MOF & yes & constant & \snr$ > 20$ & $-0.037 \pm 0.008$  \\
    \hline
    LSST  & metacal+MOF & yes & constant & \snr$ > 10$ & $-0.082 \pm 0.005$  \\
    LSST  & metacal+MOF & yes & constant & \snr$ > 15$ & $-0.067 \pm 0.005$  \\
    LSST  & metacal+MOF & yes & constant & \snr$ > 20$ & $-0.062 \pm 0.004$  \\
    \hline
    \hline
    \multicolumn{6}{c}{metadetect+moments - full scene sheared}\\
    \hline
    DES   & metadetect+moments & yes & constant & \snr$ > 10$ & $+0.00016 \pm 0.00086$  \\
    DES   & metadetect+moments & yes & constant & \snr$ > 15$ & $+0.00041 \pm 0.00079$  \\
    DES   & metadetect+moments & yes & constant & \snr$ > 20$ & $+0.00081 \pm 0.00075$  \\
    \hline
    LSST  & metadetect+moments & yes & constant & \snr$ > 10$ & $+0.00047 \pm 0.00043$  \\
    LSST  & metadetect+moments & yes & constant & \snr$ > 15$ & $-0.00002 \pm 0.00034$  \\
    LSST  & metadetect+moments & yes & constant & \snr$ > 20$ & $+0.00019 \pm 0.00028$  \\
    \hline
    \hline
    \multicolumn{6}{c}{metadetect+moments - individual objects sheared}\\
    \hline
    DES   & metadetect+moments & no & constant & \snr$ > 10$ & $-0.0036 \pm 0.0012$  \\
    DES   & metadetect+moments & no & constant & \snr$ > 15$ & $-0.0010 \pm 0.0010$  \\
    DES   & metadetect+moments & no & constant & \snr$ > 20$ & $-0.0026 \pm 0.0009$  \\
    \hline
    LSST  & metadetect+moments & no & constant & \snr$ > 10$ & $\pm$  \\
    LSST  & metadetect+moments & no & constant & \snr$ > 15$ & $\pm$  \\
    LSST  & metadetect+moments & no & constant & \snr$ > 20$ & $\pm$  \\
    \hline
  \end{tabular}

\end{table*}


\section{Mitigating Shear-dependent Detection Biases with \textsc{METACALIBRATION}}
blah

\subsection{Results for Simulated Galaxy Pairs} \label{sec:mdetpairs}

In figure \ref{fig:pairbias} we show results for pairs of galaxies, including
detection in the \mcal\ process.   The blue filled circles represent the case
where deblending is performed using MOF.  The green plus signs represent the
case where no deblending was performed. For the undeblended case we further
simplified the process:  simple weighted moments were taken at the position
determined by \sx\ using a fixed weight function with full-width at half maximum
1.2 arcsec, without any correction for the PSF.

In both cases the bias is greatly reduced, with significant bias seen only at
the special separation of 1.5 arcsec, where the two objects are detected as one
object in half of the cases.  This demonstrates that the bias we see is not
primarily due to the process of deblending itself, but rather shear-dependent
detection effects.  The remaining biases at 1.5 arcsec tend to be different
sign for the deblended and non-deblended cases, which shows there is a
qualitative difference in how the two measurements respond to the shear. As we
will show below, we find no significant bias for more realistic DES and
LSST-like images where the typical separation of galaxies is not at a special
location of maximum detection ambiguity.

\subsection{Results for Simulations with Representative Galaxy Density and Noise}
\label{sec:res:constpsf}


\subsection{Results for Simulations with Realistic Masking and PSF Variation}
\label{sec:res:varpsf}

Matt's stuff with the full glory.  If all are equally unbiased, we can remove
section \ref{sec:res:constpsf}

\mrb{Do we want to show masking?}

\subsection{Physical Limits on Shear Inference with \mdet}
\label{sec:res:physlim}


\section{Summary}
blah

\bibliographystyle{mnras}
\bibliography{references}

\appendix

\section{Fast Approximate Variable PSF Models}

In this work we use a fast, approximate variable PSF model. This model eases the
computational requirements for the simulations while also retaining the
essential features of realistic PSF variation. In this appendix, we present
the model and verify its statistical properties against more realistic PSF models.

We start with the results of \citet{heymans2012}. They fit the \vonkarman model
of atmospheric turbulence

\begin{displaymath}
  P(\ell) \propto \left(\ell^{2} + \frac{1}{\theta_{0}^2}\right)^{-11/6}
\end{displaymath}
to images with high stellar density. Here $\theta_{0}$ is the outer scale of
turbulence. \citep{heymans2012} find that $\theta_{0}\approx3$ arcmin.
We further add an additional Gaussian truncation of the power

\begin{displaymath}
  P_{trunc}(\ell) \propto P(\ell)\exp\left(-\ell^2r^{2}\right)
\end{displaymath}
at a scale of $r=1$ arcsec in order to reduce the level of resulting
PSF variation.
\esstodo{How do we justify that?   Based on exposure time?}

Using this model, we seed equal amounts of E- and B-mode power on a grid of
$128\times128$ cells using random phases. Each cell of the grid is one
arcsec in size. We normalize the overall shape variance to $0.05^2$. We then use
the $g1$ and $g2$ components of this model to set the shape of the PSF at each
location. Note that we also bound the total ellipticity to at most 0.5.
We model the PSF profile as a Moffat with shape parameter $\beta=2.5$.
The size of the Moffat profile is set to be proportional to $\mu^{-3/4}$,
where $\mu$ is the magnification computed from the power spectra realization. The
proportionality constant is drawn randomly from a log-normal model with
scatter 0.1 arcmin and a central value set so the final PSF size. This mimics
the typical seeing conditions of a given survey.

We show an example PSF for a DES-like survey in Figure~\ref{fig:pspsf}.  Over a
1 square arcminute patch, our approximate models generate PSF shape and size
variation that are $\gtrsim10\times$ that seen in real 90 second exposures
with DECam \mrb{ref}, or the expected variation in a 15 second exposure with
LSST \mrb{ref}.  Figure~\ref{fig:psxi} shows the $\xi_{\pm}$ shear correlation
functions averaged over 100 realizations of our models. For comparison, we
expect at most shear correlation function amplitudes of $\sim10^{-4}$ for LSST
\mrb{ref} and for DESCam 90 second exposures. The DECam models were generated
using the methods of \mrb{ref} but for DECam-like environmental conditions. For
the optical contributions to the PSF, we use a set of randomly drawn
aberrations (similar to GREAT3 \mrb{ref}), but with values more typical of
DECam
observations\footnote{\url{https://github.com/GalSim-developers/GalSim/blob/releases/2.1/examples/great3/cgc.yaml}}.
Finally, we note that simulations of metadetection with this PSF model show
$\approx-0.8\%$ multiplicative biases.
\esstodo{What 0.8\% for LSST depth?  Are you using smaller variations for
the LSST sims to show we are ok for LSST?}

\begin{figure*}
  \includegraphics[width=\textwidth]{figures/pspsf.pdf}
  \caption{
    Variable PSF model statistics for a DECam-like exposure. The top-left
    panel shows the variation in the FWHM in arcseconds. The top-right panel
    shows a visualization of the PSF shape variation. The bottom-left panel shows
    the variation in the $1$-component of the PSF shape. The bottom-right panel
    shows the variation in the $2$-component of the PSF shape. The variation in
    this model is $\gtrsim10\times$ larger than the typical PSF variation for
    either DECam or expected LSST observations.
    \label{fig:pspsf}}
\end{figure*}

\begin{figure}
  \includegraphics[width=\columnwidth]{figures/psxi.pdf}
  \caption{
    Variable PSF model shear correlation functions for a DECam-like exposure. LSST
    is expected to have shear correlation function magnitudes around
    $\sim10^{-4}$ \mrb{ref}.
    \label{fig:psxi}}
\end{figure}


\bsp
\label{lastpage}
\end{document}
